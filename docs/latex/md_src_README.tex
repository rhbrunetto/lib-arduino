Biblioteca para utilização de módulos básicos para Arduino

\subsection*{Introdução}

Biblioteca desenvolvida como primeiro trabalho da disciplina de Sistemas Digitais, curso Ciência da Computação pela Universidade Estadual de Maringá. O trabalho consiste em uma bilblioteca para integração com os seguintes módulos para o microcontrolador Arduino\+:


\begin{DoxyItemize}
\item Manipulação dos pinos E/S
\item Geração de ondas
\item Delay variável
\item Interface com L\+ED
\item Interface com botão
\item Interface com Display de 7-\/segmentos
\item Sensor de distância
\end{DoxyItemize}

\subsection*{Requisitos}

A biblioteca foi desenvolvida para ambiente Linux, chip alvo Atmega328p. As dependências\+:


\begin{DoxyCode}{0}
\DoxyCodeLine{avrdude}
\DoxyCodeLine{gcc-avr}
\end{DoxyCode}


\subsection*{Utilização}

Crie um arquivo fonte {\itshape main.\+c} no diretorio {\itshape src/exec}, em seguida\+: 
\begin{DoxyCode}{0}
\DoxyCodeLine{make}
\end{DoxyCode}


Para utilizar a biblioteca basta incluir as arquivos header. Ex\+:


\begin{DoxyCode}{0}
\DoxyCodeLine{\#include "pins.h"}
\end{DoxyCode}


\subsection*{Documentação}

A documentação está disponível em \href{doc/latex/refman.pdf}\texttt{ doc/latex/refman.\+pdf}.

\subsection*{Autores}

Ricardo Henrique Brunetto ra94182

Thiago Kira ra78750 